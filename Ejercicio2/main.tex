\documentclass[a4paper,12pt]{article}
\usepackage[top = 2.5cm, bottom = 2.5cm, left = 2.5cm, right = 2.5cm]{geometry}
\usepackage[T1]{fontenc}
\usepackage[utf8]{inputenc}
\usepackage{multirow} 
\usepackage{booktabs} 
\usepackage{graphicx}
\usepackage[spanish]{babel}
\usepackage{setspace}
\setlength{\parindent}{0in}
\usepackage{float}
\usepackage{fancyhdr}
\usepackage{amsmath}
\usepackage{amssymb}
\usepackage{amsthm}
\usepackage[numbers]{natbib}
\newcommand\Mycite[1]{%
	\citeauthor{#1}~[\citeyear{#1}]}
\usepackage{graphicx}
\usepackage{subcaption}
\usepackage{booktabs}
\usepackage{etoolbox}
\usepackage{minibox}
\usepackage{hyperref}
\usepackage{xcolor}
\usepackage[skins]{tcolorbox}
%---------------------------

\newtcolorbox{cajita}[1][]{
	 #1
}

\newenvironment{sol}
{\renewcommand\qedsymbol{$\square$}\begin{proof}[\textbf{Solución.}]}
	{\end{proof}}

\newenvironment{dem}
{\renewcommand\qedsymbol{$\blacksquare$}\begin{proof}[\textbf{Demostración.}]}
	{\end{proof}}

\newtheorem{problema}{Problema}
\newtheorem{definicion}{Definición}
\newtheorem{ejemplo}{Ejemplo}
\newtheorem{teorema}{Teorema}
\newtheorem{corolario}{Corolario}[teorema]
\newtheorem{lema}[teorema]{Lema}
\newtheorem{prop}{Proposición}
\newtheorem*{nota}{\textbf{NOTA}}
\renewcommand\qedsymbol{$\blacksquare$}
\usepackage{svg}
\usepackage{tikz}
\usepackage[framemethod=default]{mdframed}
\global\mdfdefinestyle{exampledefault}{%
linecolor=lightgray,linewidth=1pt,%
leftmargin=1cm,rightmargin=1cm,
}




\newenvironment{noter}[1]{%
\mdfsetup{%
frametitle={\tikz\node[fill=white,rectangle,inner sep=0pt,outer sep=0pt]{#1};},
frametitleaboveskip=-0.5\ht\strutbox,
frametitlealignment=\raggedright
}%
\begin{mdframed}[style=exampledefault]
}{\end{mdframed}}
\newcommand{\linea}{\noindent\rule{\textwidth}{3pt}}
\newcommand{\linita}{\noindent\rule{\textwidth}{1pt}}

\AtBeginEnvironment{align}{\setcounter{equation}{0}}
\pagestyle{fancy}

\fancyhf{}









%----------------------------------------------------------
\lhead{\footnotesize Teoría de Conjuntos}
\rhead{\footnotesize  Rudik Roberto Rompich}
\cfoot{\footnotesize \thepage}


%--------------------------

\begin{document}
 \thispagestyle{empty} 
    \begin{tabular}{p{15.5cm}}
    \begin{tabbing}
    \textbf{Universidad del Valle de Guatemala} \\
    Departamento de Matemática\\
    Licenciatura en Matemática Aplicada\\\\
   \textbf{Estudiante:} Rudik Roberto Rompich\\
   \textbf{Correo:}  \href{mailto:rom19857@uvg.edu.gt}{rom19857@uvg.edu.gt}\\
   \textbf{Carné:} 19857
    \end{tabbing}
    \begin{center}
        MM2033 - Teoría de Conjuntos - Catedrático: Nancy Zurita\\
        \today
    \end{center}\\
    \hline
    \\
    \end{tabular} 
    \vspace*{0.3cm} 
    \begin{center} 
    {\Large \bf  HT 6
} 
        \vspace{2mm}
    \end{center}
    \vspace{0.4cm}
%--------------------------


\begin{cajita}
	\begin{definicion}(Pareja ordenada)
		Si $\Delta \neq \square \Rightarrow(x, y)=\{\{x, \Delta\},\{y, \square\}\}$
	\end{definicion}
\end{cajita}

\begin{cajita}
	\begin{teorema}
		$\{x,y\}=\{u,v\}\implies (x=u\wedge y=v)\vee (x=v\wedge y=u)$. 
	\end{teorema}
\end{cajita}

\begin{problema}
	$(x,y)=(u,v)\implies [x=u\wedge y=v]$. 
\end{problema}
\begin{proof}
	Considerando la definición 1, $$(x,y)=(u,v)\implies \{\{x,\Delta\},\{y,\square\}\} =\{\{u,\Delta\},\{v,\square\}\}.$$
	Por teorema 1, sabemos que $$\underbrace{\left[\left(\{x,\Delta\}=\{u,\Delta\}\right)\wedge \left(\{y,\square\}=\{v,\square\}\right)\right]}_{(1)}\vee \underbrace{\left[\left(\{x,\Delta\}=\{v,\square\}\right)\wedge\left( \{y,\square\}=\{u,\Delta\}\right)\right]}_{(2)}.$$
	
	Ahora bien, analizamos el caso (1): $\left(\{x,\Delta\}=\{u,\Delta\}\right)\wedge \left(\{y,\square\}=\{v,\square\}\right)$. Aplicamos el teorema 1, nuevamente: 
	\begin{gather*}
		\left[(x=u\wedge \Delta=\Delta)\vee (x=\Delta \wedge \Delta=u)\right]\wedge\left[(y=v\wedge \square=\square)\vee (y=\square\wedge \square=v)\right] \implies\\
		\implies [(x=u)\vee (x=u=\Delta)]\wedge [(y=v)\vee (y=v=\square)] \implies \\
		\implies [x=u]\wedge [y=v]. 
	\end{gather*}

	
	El caso (2): $\left(\{x,\Delta\}=\{v,\square\}\right)\wedge\left( \{y,\square\}=\{u,\Delta\}\right)$. Aplicando el teorema 1, tenemos: 
	
	\begin{gather*}
		\left[\underbrace{(x=v\wedge \Delta=\square)}_{falso}\vee (x=\square \wedge \Delta=v)\right]\wedge\left[\underbrace{(y=u\wedge \square=\Delta)}_{falso}\vee (y=\Delta\wedge \square=u)\right]\implies\\ 
		\implies [(x=\square \wedge \Delta=v)\wedge (y=\Delta\wedge \square=u)]\implies [x=u]\wedge [y=v].
	\end{gather*}

$\therefore (x,y)=(u,v)\implies [x=u\wedge y=v]$. 

	
\end{proof}

%---------------------------
\bibliographystyle{apa}
\bibliography{referencias.bib}

\end{document}