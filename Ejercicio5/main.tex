\documentclass[a4paper,12pt]{article}
\usepackage[top = 2.5cm, bottom = 2.5cm, left = 2.5cm, right = 2.5cm]{geometry}
\usepackage[T1]{fontenc}
\usepackage[utf8]{inputenc}
\usepackage{multirow} 
\usepackage{booktabs} 
\usepackage{graphicx}
\usepackage[spanish]{babel}
\usepackage{setspace}
\setlength{\parindent}{0in}
\usepackage{float}
\usepackage{fancyhdr}
\usepackage{amsmath}
\usepackage{amssymb}
\usepackage{amsthm}
\usepackage[numbers]{natbib}
\newcommand\Mycite[1]{%
	\citeauthor{#1}~[\citeyear{#1}]}
\usepackage{graphicx}
\usepackage{subcaption}
\usepackage{booktabs}
\usepackage{etoolbox}
\usepackage{minibox}
\usepackage{hyperref}
\usepackage{xcolor}
\usepackage[skins]{tcolorbox}
%---------------------------

\newtcolorbox{cajita}[1][]{
	 #1
}

\newenvironment{sol}
{\renewcommand\qedsymbol{$\square$}\begin{proof}[\textbf{Solución.}]}
	{\end{proof}}

\newenvironment{dem}
{\renewcommand\qedsymbol{$\blacksquare$}\begin{proof}[\textbf{Demostración.}]}
	{\end{proof}}

\newtheorem{problema}{Problema}
\newtheorem{definicion}{Definición}
\newtheorem{ejemplo}{Ejemplo}
\newtheorem{teorema}{Teorema}
\newtheorem{corolario}{Corolario}[teorema]
\newtheorem{lema}[teorema]{Lema}
\newtheorem{prop}{Proposición}
\newtheorem*{nota}{\textbf{NOTA}}
\renewcommand\qedsymbol{$\blacksquare$}
\usepackage{svg}
\usepackage{tikz}
\usepackage[framemethod=default]{mdframed}
\global\mdfdefinestyle{exampledefault}{%
linecolor=lightgray,linewidth=1pt,%
leftmargin=1cm,rightmargin=1cm,
}




\newenvironment{noter}[1]{%
\mdfsetup{%
frametitle={\tikz\node[fill=white,rectangle,inner sep=0pt,outer sep=0pt]{#1};},
frametitleaboveskip=-0.5\ht\strutbox,
frametitlealignment=\raggedright
}%
\begin{mdframed}[style=exampledefault]
}{\end{mdframed}}
\newcommand{\linea}{\noindent\rule{\textwidth}{3pt}}
\newcommand{\linita}{\noindent\rule{\textwidth}{1pt}}

\AtBeginEnvironment{align}{\setcounter{equation}{0}}
\pagestyle{fancy}

\fancyhf{}









%----------------------------------------------------------
\lhead{\footnotesize Teoría de Conjuntos}
\rhead{\footnotesize  Rudik Roberto Rompich}
\cfoot{\footnotesize \thepage}


%--------------------------

\begin{document}
 \thispagestyle{empty} 
    \begin{tabular}{p{15.5cm}}
    \begin{tabbing}
    \textbf{Universidad del Valle de Guatemala} \\
    Departamento de Matemática\\
    Licenciatura en Matemática Aplicada\\\\
   \textbf{Estudiante:} Rudik Roberto Rompich\\
   \textbf{Correo:}  \href{mailto:rom19857@uvg.edu.gt}{rom19857@uvg.edu.gt}\\
   \textbf{Carné:} 19857
    \end{tabbing}
    \begin{center}
        MM2033 - Teoría de Conjuntos - Catedrático: Nancy Zurita\\
        \today
    \end{center}\\
    \hline
    \\
    \end{tabular} 
    \vspace*{0.3cm} 
    \begin{center} 
    {\Large \bf  HT 6
} 
        \vspace{2mm}
    \end{center}
    \vspace{0.4cm}
%--------------------------


\begin{problema}
	Si $A$ es un conjunto entonces existe una función biyectiva entre $2^A$ y $\mathcal{P}(A)$. 
\end{problema}
\begin{cajita}
	\begin{definicion}(Función característica)
		Sea $A$ un conjunto  y $B\subseteq A$ entonces se define la función característica de $B$ de la siguiente manera: 
		$$C_B:A\to 2\ni $$
		$$C_B(x)=\begin{cases}
			0, & x \in B\\
			1, & x \not\in B
		\end{cases}, \quad x\in A.$$
	\end{definicion}
\end{cajita}

\begin{proof}
Sea $\gamma: \mathcal{P}(A)\to 2^A$ y supóngase $B\in \mathcal{P}(A)$,  tal que

 $$\gamma(B)=\underbrace{\{C_B\}_{B\in A}}_{\text{función}\atop \text{característica}}, \quad \forall B\in \mathcal{P}(A).$$


 Nótese que $\{C_B\}_{B\in A}  \subseteq 2^A$. Comprobaremos que $\gamma$ es una función y posteriormente que es una función biyectiva:
\begin{enumerate}
	\item Función, tal que por el teorema de caracterización para funciones: \begin{enumerate}
		\item $B\in \mathcal{P}(A) \ni \gamma(B)=\{C_B\}_{B\in A}$. 
		\item Inyectividad en el inciso \textbf{2.a}. 
	\end{enumerate}
	\item Biyectividad, tal que 
	\begin{enumerate}
		\item Inyectividad. Supóngase $X$ y $Y\in \mathcal{P}(A)\ni$
		\begin{align*}
			 \gamma (X) &=\gamma(Y)\\
			 \{C_X\}_{X\in A}&=\{C_Y\}_{Y\in A}\\
			 \{x\in A\ni C_X(x)=0\} &=\{x\in A\ni C_Y(x)=0\}\\
			 X&=Y
		\end{align*}
	Por lo tanto, $\gamma$ es inyectiva. 
		\item Sobreyectividad. Supóngase que tenemos una función $f\in 2^A$ y ahora dígase que $f^{-1}(0)=B\implies f=\{C_B\}_{B\in A}=\gamma(B)$. Por lo tanto, $\gamma$ es sobreyectiva. 
	\end{enumerate}
Por lo tanto, $\gamma$ es función biyectiva. 
\end{enumerate}	
Entonces, $2^A \subseteq \{C_B\}_{B\in A} $ tal que $2^A = \{C_B\}_{B\in A} $. Por lo tanto, existe una función $\gamma$ biyectiva entre $2^A$ y $\mathcal{P}(A)$. 
\end{proof}



%---------------------------
\bibliographystyle{apa}
\bibliography{referencias.bib}

\end{document}