\documentclass[a4paper,12pt]{article}
\usepackage[top = 2.5cm, bottom = 2.5cm, left = 2.5cm, right = 2.5cm]{geometry}
\usepackage[T1]{fontenc}
\usepackage[utf8]{inputenc}
\usepackage{multirow} 
\usepackage{booktabs} 
\usepackage{graphicx}
\usepackage[spanish]{babel}
\usepackage{setspace}
\setlength{\parindent}{0in}
\usepackage{float}
\usepackage{fancyhdr}
\usepackage{amsmath}
\usepackage{amssymb}
\usepackage{amsthm}
\usepackage[numbers]{natbib}
\newcommand\Mycite[1]{%
	\citeauthor{#1}~[\citeyear{#1}]}
\usepackage{graphicx}
\usepackage{subcaption}
\usepackage{booktabs}
\usepackage{etoolbox}
\usepackage{minibox}
\usepackage{hyperref}
\usepackage{xcolor}
\usepackage[skins]{tcolorbox}
%---------------------------

\newtcolorbox{cajita}[1][]{
	 #1
}

\newenvironment{sol}
{\renewcommand\qedsymbol{$\square$}\begin{proof}[\textbf{Solución.}]}
	{\end{proof}}

\newenvironment{dem}
{\renewcommand\qedsymbol{$\blacksquare$}\begin{proof}[\textbf{Demostración.}]}
	{\end{proof}}

\newtheorem{problema}{Problema}
\newtheorem{definicion}{Definición}
\newtheorem{ejemplo}{Ejemplo}
\newtheorem{teorema}{Teorema}
\newtheorem{corolario}{Corolario}[teorema]
\newtheorem{lema}[teorema]{Lema}
\newtheorem{prop}{Proposición}
\newtheorem*{nota}{\textbf{NOTA}}
\renewcommand\qedsymbol{$\blacksquare$}
\usepackage{svg}
\usepackage{tikz}
\usepackage[framemethod=default]{mdframed}
\global\mdfdefinestyle{exampledefault}{%
linecolor=lightgray,linewidth=1pt,%
leftmargin=1cm,rightmargin=1cm,
}




\newenvironment{noter}[1]{%
\mdfsetup{%
frametitle={\tikz\node[fill=white,rectangle,inner sep=0pt,outer sep=0pt]{#1};},
frametitleaboveskip=-0.5\ht\strutbox,
frametitlealignment=\raggedright
}%
\begin{mdframed}[style=exampledefault]
}{\end{mdframed}}
\newcommand{\linea}{\noindent\rule{\textwidth}{3pt}}
\newcommand{\linita}{\noindent\rule{\textwidth}{1pt}}

\AtBeginEnvironment{align}{\setcounter{equation}{0}}
\pagestyle{fancy}

\fancyhf{}









%----------------------------------------------------------
\lhead{\footnotesize Teoría de Conjuntos}
\rhead{\footnotesize  Rudik Roberto Rompich}
\cfoot{\footnotesize \thepage}


%--------------------------

\begin{document}
 \thispagestyle{empty} 
    \begin{tabular}{p{15.5cm}}
    \begin{tabbing}
    \textbf{Universidad del Valle de Guatemala} \\
    Departamento de Matemática\\
    Licenciatura en Matemática Aplicada\\\\
   \textbf{Estudiante:} Rudik Roberto Rompich\\
   \textbf{Correo:}  \href{mailto:rom19857@uvg.edu.gt}{rom19857@uvg.edu.gt}\\
   \textbf{Carné:} 19857
    \end{tabbing}
    \begin{center}
        MM2033 - Teoría de Conjuntos - Catedrático: Nancy Zurita\\
        \today
    \end{center}\\
    \hline
    \\
    \end{tabular} 
    \vspace*{0.3cm} 
    \begin{center} 
    {\Large \bf  Ejercicio 6
} 
        \vspace{2mm}
    \end{center}
    \vspace{0.4cm}
%--------------------------
\section{Problema}

\begin{problema}
	Sea $f:A\to B$ una función y $G$ la relación definida por $f$. Se definen entonces las siguientes funciones:
	\begin{enumerate}
		\item $r:A\to A/G\ni r(x)=[x]$
		\item $s:A/G\to\bar{f}(A)\ni s([x])=f(x), \forall x\in A$
		\item $t:\bar{f}(A)\to B\ni t(y)=y,\forall y\in A$
	\end{enumerate}
\begin{cajita}
	$$\bar{f}(A)=\{y\in B | \exists x \in A \ni y=f(x)\}$$
\end{cajita}
	 Demostrar:
	\begin{enumerate}
		\item $r$ es sobreyectiva. 
			\begin{dem}
				Por la definición de sobreyectividad, $$\forall [x]\in A/G, \exists x \in A \ni r(x)=[x].$$
			\end{dem}
		\item $s$ biyectiva.
			\begin{dem}
				A probar: $s$ es sobreyectiva e inyectiva. 
				\begin{enumerate}
					\item Sobreyectiva.  Por la definición, 
					$$\forall f(x)\in \bar{f}(A),\ \exists x\in A, \ \exists [x]\in A/G\ni s([x])=f(x).$$
					\item Inyectiva. Sean $x,y\in A\ni$
					\begin{align*}
						s([x]) &= s([y])\\
						f(x) &= f(y)
						\intertext{Como $(x,y)\in G$}
						[x] &= [y].
					\end{align*}
				\end{enumerate}
			Por lo tanto, $s$ es biyectiva. 
			\end{dem}
		\item $t$ inyectiva.
			\begin{dem}
				Sean $x,y\in A\ni$ 
				\begin{align*}
					t(x)&=t(y)\\
					x&=y
				\end{align*}
			Por lo tanto, $t$ es inyectiva. 
					\end{dem}
		\item $f=t\circ s\circ r$
			\begin{dem}
				Sea $x\in A\ni $
				$$t\left(s\left(r(x)\right)\right)\implies t\left(s\left(\left[x\right]\right)\right)\implies t(f(x))\implies f(x).$$
				Entonces, $$\left[t\circ s\circ r\right](x)=f(x),\forall x\in A $$
				Por el teorema 2.5 de Pinter podemos concluir que 
				$$\left[t\circ s\circ r\right]=f.$$
			\end{dem}
	\end{enumerate}
\end{problema}
	

%---------------------------
\bibliographystyle{apa}
\bibliography{referencias.bib}

\end{document}