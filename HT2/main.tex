\documentclass[a4paper,12pt]{article}
\usepackage[top = 2.5cm, bottom = 2.5cm, left = 2.5cm, right = 2.5cm]{geometry}
\usepackage[T1]{fontenc}
\usepackage[utf8]{inputenc}
\usepackage{multirow} 
\usepackage{booktabs} 
\usepackage{graphicx}
\usepackage[spanish]{babel}
\usepackage{setspace}
\setlength{\parindent}{0in}
\usepackage{float}
\usepackage{fancyhdr}
\usepackage{amsmath}
\usepackage{amssymb}
\usepackage{amsthm}
\usepackage[numbers]{natbib}
\newcommand\Mycite[1]{%
	\citeauthor{#1}~[\citeyear{#1}]}
\usepackage{graphicx}
\usepackage{subcaption}
\usepackage{booktabs}
\usepackage{etoolbox}
\usepackage{minibox}
\usepackage{hyperref}
\usepackage{xcolor}
\usepackage[skins]{tcolorbox}
%---------------------------

\newtcolorbox{cajita}[1][]{
	 #1
}

\newenvironment{sol}
{\renewcommand\qedsymbol{$\square$}\begin{proof}[\textbf{Solución.}]}
	{\end{proof}}

\newenvironment{dem}
{\renewcommand\qedsymbol{$\blacksquare$}\begin{proof}[\textbf{Demostración.}]}
	{\end{proof}}

\newtheorem{problema}{Problema}
\newtheorem{definicion}{Definición}
\newtheorem{ejemplo}{Ejemplo}
\newtheorem{teorema}{Teorema}
\newtheorem{corolario}{Corolario}[teorema]
\newtheorem{lema}[teorema]{Lema}
\newtheorem{prop}{Proposición}
\newtheorem*{nota}{\textbf{NOTA}}
\renewcommand\qedsymbol{$\blacksquare$}
\usepackage{svg}
\usepackage{tikz}
\usepackage[framemethod=default]{mdframed}
\global\mdfdefinestyle{exampledefault}{%
linecolor=lightgray,linewidth=1pt,%
leftmargin=1cm,rightmargin=1cm,
}




\newenvironment{noter}[1]{%
\mdfsetup{%
frametitle={\tikz\node[fill=white,rectangle,inner sep=0pt,outer sep=0pt]{#1};},
frametitleaboveskip=-0.5\ht\strutbox,
frametitlealignment=\raggedright
}%
\begin{mdframed}[style=exampledefault]
}{\end{mdframed}}
\newcommand{\linea}{\noindent\rule{\textwidth}{3pt}}
\newcommand{\linita}{\noindent\rule{\textwidth}{1pt}}

\AtBeginEnvironment{align}{\setcounter{equation}{0}}
\pagestyle{fancy}

\fancyhf{}









%----------------------------------------------------------
\lhead{\footnotesize Teoría de Conjuntos}
\rhead{\footnotesize  Rudik Roberto Rompich}
\cfoot{\footnotesize \thepage}


%--------------------------

\begin{document}
 \thispagestyle{empty} 
    \begin{tabular}{p{15.5cm}}
    \begin{tabbing}
    \textbf{Universidad del Valle de Guatemala} \\
    Departamento de Matemática\\
    Licenciatura en Matemática Aplicada\\\\
   \textbf{Estudiante:} Rudik Roberto Rompich\\
   \textbf{Correo:}  \href{mailto:rom19857@uvg.edu.gt}{rom19857@uvg.edu.gt}\\
   \textbf{Carné:} 19857
    \end{tabbing}
    \begin{center}
        MM2033 - Teoría de Conjuntos - Catedrático: Nancy Zurita\\
        \today
    \end{center}\\
    \hline
    \\
    \end{tabular} 
    \vspace*{0.3cm} 
    \begin{center} 
    {\Large \bf  HT 6
} 
        \vspace{2mm}
    \end{center}
    \vspace{0.4cm}
%--------------------------

	
Para A y B conjuntos:
\begin{enumerate}
	\item $(A \cap B) \subseteq A$
	\begin{proof}
		$x\in A\cap B\implies [x\in A \wedge x \in B]\implies x\in A$. Por axioma de extensión $(A \cap B) \subseteq A$. 
	\end{proof}
	\item Si $A \subseteq B$ entonces 
	\begin{enumerate}
		\item  $(A \cap B)=A$
		\begin{proof}
			Por hipótesis, $A\subseteq B$, $x\in A\implies x\in B$. Entonces $x\in A\cap B\iff (\underbrace{x\in A}_{\implies x\in B} \wedge x \in B)\iff x\in A$. Por axioma de extensión $(A \cap B)=A$. 
		\end{proof}
		\item $(A \cup B)=B$
		\begin{proof}
			Por hipótesis, $A\subseteq B$, $x\in A\implies x\in B$. Entonces $x\in A\cup B\iff (\underbrace{x\in A}_{\implies x\in B}\vee x\in B)\iff x\in B$. Por axioma de extensión, $(A \cup B)=B$. 
		\end{proof}
	\end{enumerate}
	\item $A \subseteq(A \cup B)$
	\begin{proof}
		$x\in A\implies (x\in A\vee x \in B)\implies x\in A\cup B$. Por axioma de extensión, $A \subseteq(A \cup B)$. 
	\end{proof}
	\item Si $A \subseteq C$ y $B \subseteq C$ entonces $(A \cup B) \subseteq C$.
	\begin{proof}
		Por hipótesis, $[x\in A\implies x\in C]\wedge [x\in B\implies x\in C]$. Sea $x\in A\cup B$. Entonces $(\underbrace{x\in A}_{\implies x\in C}\vee \underbrace{x\in B}_{\implies x\in C})$, por hipótesis $x\in C$. Por axioma de extensión, $(A \cup B) \subseteq C$. 
	\end{proof}
	\item $ A \cap(A \sim B)=A \sim B$
	\begin{proof}
		$x\in [A\cap (A\sim B)]\iff x\in A  \wedge[ x \in (A\sim B)]\iff x\in A \wedge [x\in A \wedge x\not\in B]\iff (x\in A)\wedge (x\not\in B)$. Por axioma de extensión, $ A \cap(A \sim B)=A \sim B$. 
	\end{proof}
	\item $(A \cap B) \sim B=\varnothing$
	\begin{proof}
		$x\in [(A\cap B)\sim B]\iff [x\in (A\cap B)]\wedge [x\not\in B]\iff [x\in A \wedge x\in B]\wedge [x\not\in B]\iff x\in A \wedge (x\in B\wedge x\not\in B)$ ($\to\gets$). $\forall x,  x\not\in [(A \cap B) \sim B]\implies (A \cap B) \sim B=\varnothing$. 
	\end{proof}
	\item $(A \sim B) \cap B=\varnothing$
	\begin{proof}
		$x\in [(A \sim B) \cap B]\iff [x\in (A\sim B)\wedge x\in B]\iff [x\in A \wedge x\not\in B\wedge x\in B] (\to \gets)$. $\forall x, x\not\in [(A \sim B) \cap B] \implies (A \sim B) \cap B=\varnothing$. 
	\end{proof}
	\item Si se define la diferencia simétrica como $$A \div B=(A \sim B) \cup(B \sim A)$$.
	Entonces demuestre que
	\begin{enumerate}
		\item $(A \div \varnothing)=A$
		\begin{proof}
			Usamos la definición de diferencia simétrica, $(A \div \varnothing)=(A\sim \varnothing)\cup (\varnothing\sim A)$. Entonces, $x\in [(A\sim \varnothing)\cup (\varnothing\sim A)]\iff [x\in (A\sim \varnothing) \vee x\in  (\varnothing\sim A)]\iff [x\in A \wedge x \not\in \varnothing]\vee [x\in \varnothing \wedge x\not\in A]$. Si (1) $x\in A \wedge x \not\in \varnothing\implies x\in A$, y (2) $[x\in \varnothing \wedge x\not\in A]$ ambos son falsos, entonces $x\in A$. Por axioma de extensión, $(A \div \varnothing)=A$. 
		\end{proof}
		\item  Si $A=B$ entonces $A \div B=\varnothing$
		\begin{proof}
			Por hipótesis, $x\in A\iff x\in B$. Por la definición de diferencia simétrica $A\div B= (A \sim B) \cup(B \sim A)$. Entonces $x\in [(A \sim B) \cup(B \sim A)]\iff [x\in (A \sim B)]\vee [x\in (B \sim A)]\iff [x\in A\wedge x\not\in B]\vee [x\in B\wedge x\not\in A] $. Por hipótesis, $[x\in A\wedge x\not\in A]\vee [x\in B\wedge x\not\in B] (\to\gets)$. $\forall x, x\not\in (A\div B) \implies A \div B=\varnothing$. 
		\end{proof}
	\end{enumerate}
\end{enumerate}


%---------------------------
\bibliographystyle{apa}
\bibliography{referencias.bib}

\end{document}