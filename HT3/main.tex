\documentclass[a4paper,12pt]{article}
\usepackage[top = 2.5cm, bottom = 2.5cm, left = 2.5cm, right = 2.5cm]{geometry}
\usepackage[T1]{fontenc}
\usepackage[utf8]{inputenc}
\usepackage{multirow} 
\usepackage{booktabs} 
\usepackage{graphicx}
\usepackage[spanish]{babel}
\usepackage{setspace}
\setlength{\parindent}{0in}
\usepackage{float}
\usepackage{fancyhdr}
\usepackage{amsmath}
\usepackage{amssymb}
\usepackage{amsthm}
\usepackage[numbers]{natbib}
\newcommand\Mycite[1]{%
	\citeauthor{#1}~[\citeyear{#1}]}
\usepackage{graphicx}
\usepackage{subcaption}
\usepackage{booktabs}
\usepackage{etoolbox}
\usepackage{minibox}
\usepackage{hyperref}
\usepackage{xcolor}
\usepackage[skins]{tcolorbox}
%---------------------------

\newtcolorbox{cajita}[1][]{
	 #1
}

\newenvironment{sol}
{\renewcommand\qedsymbol{$\square$}\begin{proof}[\textbf{Solución.}]}
	{\end{proof}}

\newenvironment{dem}
{\renewcommand\qedsymbol{$\blacksquare$}\begin{proof}[\textbf{Demostración.}]}
	{\end{proof}}

\newtheorem{problema}{Problema}
\newtheorem{definicion}{Definición}
\newtheorem{ejemplo}{Ejemplo}
\newtheorem{teorema}{Teorema}
\newtheorem{corolario}{Corolario}[teorema]
\newtheorem{lema}[teorema]{Lema}
\newtheorem{prop}{Proposición}
\newtheorem*{nota}{\textbf{NOTA}}
\renewcommand\qedsymbol{$\blacksquare$}
\usepackage{svg}
\usepackage{tikz}
\usepackage[framemethod=default]{mdframed}
\global\mdfdefinestyle{exampledefault}{%
linecolor=lightgray,linewidth=1pt,%
leftmargin=1cm,rightmargin=1cm,
}




\newenvironment{noter}[1]{%
\mdfsetup{%
frametitle={\tikz\node[fill=white,rectangle,inner sep=0pt,outer sep=0pt]{#1};},
frametitleaboveskip=-0.5\ht\strutbox,
frametitlealignment=\raggedright
}%
\begin{mdframed}[style=exampledefault]
}{\end{mdframed}}
\newcommand{\linea}{\noindent\rule{\textwidth}{3pt}}
\newcommand{\linita}{\noindent\rule{\textwidth}{1pt}}

\AtBeginEnvironment{align}{\setcounter{equation}{0}}
\pagestyle{fancy}

\fancyhf{}









%----------------------------------------------------------
\lhead{\footnotesize Teoría de Conjuntos}
\rhead{\footnotesize  Rudik Roberto Rompich}
\cfoot{\footnotesize \thepage}


%--------------------------

\begin{document}
 \thispagestyle{empty} 
    \begin{tabular}{p{15.5cm}}
    \begin{tabbing}
    \textbf{Universidad del Valle de Guatemala} \\
    Departamento de Matemática\\
    Licenciatura en Matemática Aplicada\\\\
   \textbf{Estudiante:} Rudik Roberto Rompich\\
   \textbf{Correo:}  \href{mailto:rom19857@uvg.edu.gt}{rom19857@uvg.edu.gt}\\
   \textbf{Carné:} 19857
    \end{tabbing}
    \begin{center}
        MM2033 - Teoría de Conjuntos - Catedrático: Nancy Zurita\\
        \today
    \end{center}\\
    \hline
    \\
    \end{tabular} 
    \vspace*{0.3cm} 
    \begin{center} 
    {\Large \bf  Ejercicio 6
} 
        \vspace{2mm}
    \end{center}
    \vspace{0.4cm}
%--------------------------

\begin{enumerate}
	\item Leer las secciones 2.7 (Axioma del conjunto Potencia) y 2.8 (Producto Cartesiano entre conjuntos)
	\item Prepararse para presentar dichos ejercicios en clase el día jueves 30 de Julio .
\end{enumerate}
\section{Página 32}
	Resolver todos los ejercicios de la página 32 (Total 6).
	
	\begin{problema}
		Hallar: $\mathcal{P}\{\text{Arquímedes} \}, \mathcal{P}\mathcal{P}\{\text{Arquímedes} \}, \mathcal{P}\{\{\text{Arquímedes}, Newton \},\varnothing \}$.
	\end{problema}
	\begin{sol}
		Tenemos: 
		\begin{enumerate}
			\item $\mathcal{P}\{\text{Arquímedes} \}=\{\varnothing, \text{Arquímedes} \}$
			\item $\mathcal{P}\mathcal{P}\{\text{Arquímedes} \}=\{\varnothing,\{\varnothing\},\{\text{Arquímedes} \},  \{\varnothing, \text{Arquímedes} \}\}$
			\item $\mathcal{P}\{\{\text{Arquímedes}, Newton \},\varnothing \}=\{\varnothing,\{\{\text{Arquímedes}, Newton \}\}, \{\varnothing\}, \{\text{Arquímedes}, Newton \}\}$
		\end{enumerate}
	\end{sol}

%------

	\begin{problema}
	Hallar: $\mathcal{P}\mathcal{P}\mathcal{P}\varnothing$.
\end{problema}
\begin{sol}
	Considerando $\mathcal{P}\varnothing =\{\varnothing\}$. 
	$$\implies \mathcal{P}\mathcal{P}\varnothing= \{\varnothing,\{\varnothing\}\}\implies \mathcal{P}\mathcal{P}\mathcal{P}\varnothing= \{\varnothing,\{\varnothing\},\{\{\varnothing\}\}, \{\varnothing,\{\varnothing\}\} \}. $$
	Supóngase que $\exists B\neq 0 \ni B\in \mathcal{P}\mathcal{P}\mathcal{P}\varnothing\implies  B\subseteq\mathcal{P}\mathcal{P}\varnothing \ (x\in B\implies x\in \{\varnothing,\{\varnothing\}\}\implies x=\varnothing \vee x=\{\varnothing\})\implies B=\{\varnothing\}\vee B=\{\{\varnothing\}\}\vee B=\{\varnothing,\{\varnothing\}\}$. Por lo tanto, $\mathcal{P}\mathcal{P}\mathcal{P}\varnothing= \{\varnothing,\{\varnothing\},\{\{\varnothing\}\}, \{\varnothing,\{\varnothing\}\} \}$. 
\end{sol}

%------

	\begin{problema}
	Demostrar los teoremas 88 y 90. 
	\begin{enumerate}
		\item[88.] $\varnothing\in \mathcal{P}A$.
		\begin{proof}
			Previamente, conocíamos que $\varnothing \subseteq A$. $\therefore$ Por teorema de caracterización, $$\varnothing \subseteq A \implies \varnothing\in \mathcal{P}A.$$		\end{proof}
		
		\item[90.] $\mathcal{P}\mathcal{P}\varnothing=\{\varnothing,\{\varnothing\}\}$.
		\begin{proof}
			Supóngase $\exists B\neq 0\ni B\in \mathcal{P}\mathcal{P}\varnothing\iff B\subseteq \mathcal{P}\varnothing \ (x\in B\implies x\in \{\varnothing\}\implies x\in \varnothing)\implies B=\{\varnothing\}$. Por lo tanto, $\mathcal{P}\mathcal{P}\varnothing=\{\varnothing,\{\varnothing\}\}$.
		\end{proof}
	\end{enumerate}
\end{problema}


%------
	\begin{problema}
	Demostrar el teorema 93. $\mathcal{P}(A\cap B)=(\mathcal{P} A)\cap (\mathcal{P}B)$. 
\end{problema}
\begin{proof}
	Sean $A,B$ y $C$ conjuntos. Supóngase
	$C\in \mathcal{P}(A\cap B) \iff C\subseteq (A\cap B)\ (x\in C\implies x\in A\wedge x\in B) \iff C\subseteq  A \wedge C\subseteq  B \iff C\in   \mathcal{P}A \wedge C\in  \mathcal{P} B\iff  C\in [(\mathcal{P} A)\cap (\mathcal{P}B)]. \therefore \ \mathcal{P}(A\cap B)=(\mathcal{P} A)\cap (\mathcal{P}B)$.
\end{proof}

%------
	\begin{problema}
	Demostrar teorema 94. $\mathcal{P}(A \sim B)\subseteq ((\mathcal{P}A)\sim(\mathcal{P}B) )\cup \{\varnothing\}$.
\end{problema}
\begin{proof}
	
	Sean $A,B$ y $C$ conjuntos, tal que $C\in \mathcal{P}(A\sim B)$ 
	\begin{enumerate}
		\item $A\sim B$ no es vacío. $C\in \mathcal{P}(A \sim B) \implies C\subseteq (A\sim B)\ (x\in C\implies x\in A \wedge x\not\in B\implies [\forall x\in C\implies x\in A]\wedge [x\in C\implies x\not\in B])\implies C\subseteq A \wedge C\not\subseteq B\implies C\in \mathcal{P}A \wedge C\not\in \mathcal{P}B\implies C\in [(\mathcal{P}A) \sim (\mathcal{P} B)]\implies \mathcal{P}(A \sim B)\subseteq ((\mathcal{P}A)\sim(\mathcal{P}B) )$.
		\item $A\sim B$ es vacío. $\implies C\in \mathcal{P}(A\sim B)= C\in \{\varnothing\}\implies C=0\implies \mathcal{P}(A\sim B)\subseteq \{\varnothing\}$.
	\end{enumerate}
	
	 Por lo tanto, $\mathcal{P}(A \sim B)\subseteq ((\mathcal{P}A)\sim(\mathcal{P}B) )\cup \{\varnothing\}$.
\end{proof}

%------
	\begin{problema}
	Dar contraejemplos para mostrar que no siempre es el caso de que 
	\begin{enumerate}
		\item $(\mathcal{P}A)\cup (\mathcal{P}B)=\mathcal{P}(A\cup B)$.
		\begin{sol}
			Supóngase que tenemos un conjunto $A=\{0,1\}$ y $B=\{2\}$. 
			\begin{enumerate}
				\item $(\mathcal{P}A)\cup (\mathcal{P}B)$. Implica $\mathcal{P}A=\{0,\{0\}, \{1\}, \{0,1\}\}$, $\mathcal{P}B=\{0,\{2\}\}$. Entonces, $$(\mathcal{P}A)\cup (\mathcal{P}B)= \{0,\{0\}, \{1\}, \{0,1\}, \{2\}\}.$$
				\item $\mathcal{P}(A\cup B)$. Implica $A\cup B=\{0,1,2\}$. Entonces $$\mathcal{P}(A\cup B)= \{0, \{0\}, \{1\},\{2\},\{0,1\},\{0,2\},\{1,2\}, \{0,1,2\}\}$$
			\end{enumerate}
		Por lo tanto, $(\mathcal{P}A)\cup (\mathcal{P}B)\neq \mathcal{P}(A\cup B)$
		\end{sol}
		\item $\mathcal{P}(A \sim B)= (\mathcal{P}A)\sim (\mathcal{P}B)$.
		\begin{sol}
			Supóngase que $A=\{2,1\}$ y $B=\{2\}$. 
			\begin{enumerate}
				\item $\mathcal{P}(A \sim B)$. Implica $A\sim B= \{1\}$. Entonces $\mathcal{P}(A \sim B)=\{0,\{1\}\}$.
				\item $(\mathcal{P}A)\sim (\mathcal{P}B)$. Implica $(\mathcal{P}A)=\{0,\{2\},\{1\},\{2,1\} \}$ y $(\mathcal{P}B)=\{0,\{2\}\}$. Entonces: 
				$$(\mathcal{P}A)\sim (\mathcal{P}B)=\{\{1\},\{2,1\}\}.$$
			\end{enumerate}
		Por lo tanto, 
		$$\mathcal{P}(A \sim B)\neq (\mathcal{P}A)\sim (\mathcal{P}B).$$
		\end{sol}
	\end{enumerate}
\end{problema}


%------
\section{Página 34}	
	Resolver todos los ejercicios de la página 34 (Total 6)
	
	\begin{problema}
	 Demostrar los teoremas 96 y 97 .\begin{enumerate}
	 	\item[96.] $x\in A\times B\iff (\exists y)(\exists z)(y\in A\wedge z\in B\wedge x=\langle y, z\rangle)$.
	 	\begin{proof}
	 		Sean $A,B$ y $C$ conjuntos. Inmediatamente por el teorema de existencia. Sea $A\times B = A\times B$, entonces 
	 		$\forall x\in A\times B\iff (\exists y)(\exists z)(y\in A\wedge z\in B\wedge x=\langle y, z\rangle)$.
	 	\end{proof}
	 	
	 	\item[97.] $\langle x,y\rangle \in A\times B \iff x\in A \wedge y\in B$.
	 	\begin{proof}
	 		Sean $A,B$ y $C$ conjuntos. Inmediatamente por la definición. Sea $A\times B=\{\langle x,y\rangle: x\in A\wedge y\in B\}$. Por lo tanto, $\langle x,y\rangle \in A\times B \iff x\in A \wedge y\in B$.
	 	\end{proof}
	 	
	 \end{enumerate}
\end{problema}

%------

\begin{problema}
	Demostrar el teorema 101. $B\subseteq C\implies A\times B\subseteq A\times C$.
\end{problema}
\begin{proof}
	Sean $A,B$ y $C$ conjuntos. Por hipótesis, $B\subseteq C\iff (y\in B\implies y\in C)$. Entonces, supóngase que $\langle x,y\rangle \in A\times B\implies x\in A\wedge y\in B\implies x\in A\wedge y\in C\implies\langle x,y\rangle \in A\times C $. $\therefore A\times B\subseteq A\times C$. 
\end{proof}

%------

\begin{problema}
	Demostrar los teoremas 103  y 104.
	\begin{enumerate}
		\item[103.] $A\times (B\cup C)=(A\times B)\cup (A\times C)$.
		\begin{proof}
			Sean $A,B$ y $C$ conjuntos. Supóngase $\langle x, y\rangle \in \left[A\times (B\cup C)\right]\iff x\in A\wedge [y\in (B\cup C)]\iff x\in A \wedge (y\in B\vee y\in C)\iff (x\in A\wedge y\in B)\vee(x\in A\wedge y\in C)\iff \langle x, y \rangle \in (A\times B)\cup (A\times C)$. $\therefore A\times (B\cup C)=(A\times B)\cup (A\times C)$.  
		\end{proof}
		\item[104.] $A\times (B \sim C)= (A\times B)\sim (A\times C)$.
			\begin{cajita}
			Nótese que $\langle x,y \rangle \not\in A\times B \iff x\not\in A \vee x\not\in B  $.
		\end{cajita}
	\begin{proof}
			Sean $A,B$ y $C$ conjuntos. $\langle x, y\rangle \in A\times (B\sim C)\iff x\in A \wedge \left[y\in (B\sim C)\right]\iff x\in A \wedge [(y\in B)\wedge (y\not\in C)]\iff [(x\in A) \wedge (y\in B)] \wedge  ( y\not\in C)\iff \langle x, y\rangle \in (A\times B)\sim (A\times C) $. 
	\end{proof}
	\end{enumerate}
\end{problema}


%------
\begin{problema}
	Dar un contra-ejemplo simple para mostrar que, en general, no es el caso de que
	$$
	A \cup(B \times C)=(A \times B) \cup(A \times C)
	$$
\end{problema}
\begin{proof}
	Supóngase $A=\{1\}, B=\{2\}, C=\{3\}$. Comprobamos:
	\begin{enumerate}
		\item $A \cup(B \times C)$.  Determinamos, $(B\times C)=\{\langle 2,3\rangle\}$ tal que $A \cup(B \times C)=\{1,\langle 2,3\rangle \}$
		\item $(A \times B) \cup(A \times C)$. Determinamos, $(A \times B) = \{\langle 1, 2\rangle \}$ y $(A\times C)=\{\langle 1, 3\rangle\}$. Entonces, $(A \times B) \cup(A \times C)= \{\langle 1, 2\rangle, \langle 1, 3\rangle \}$
	\end{enumerate}
	Por lo tanto, $$
	A \cup(B \times C)\neq (A \times B) \cup(A \times C)
	$$
\end{proof}

%------
\begin{problema}
	¿Es asociativa la operación producto cartesiano?  Si lo es, demostrarlo. Si no, dar un contra-ejemplo.
\end{problema}
\begin{sol}
	Procedemos con un contraejemplo para deducir que $$A\times(B\times C)\neq(A\times B)\times C$$
	Sea $A=\{1\}, B=\{2\}, C=\{3\}$, entonces 
	\begin{enumerate}
		\item $A\times(B\times C)$. Determinamos $(B\times C)=\{\langle 2,3\rangle \}$. Por lo tanto, $A\times(B\times C)= \{\langle 1, \langle 2,3\rangle \rangle \} $. 
		\item $(A\times B)\times C$. Determinamos $(A\times B)=\{\langle 1,2\rangle\}$. Por lo tanto, $(A\times B)\times C= \{\langle \langle 1,2\rangle, 3\rangle \}$
	\end{enumerate}
Por lo tanto, no es asociativa.
\end{sol}

%------
\begin{problema}
	Demostrar que
	$$
	A \times \bigcap B=\bigcap_{C \in B}(A \times C) \textcolor{red}{=\bigcap (A\times B)}
	$$
\end{problema}
\begin{cajita}
	Según Suppes, 
	$$\bigcup A= \bigcup_{B\in A}B\quad \text{ y }\quad \bigcap A=\bigcap_{B\in A}B $$
\end{cajita}
\begin{proof}
	$\langle x,y\rangle \in A\times \bigcap B\iff x\in A \wedge y\in \bigcap B\iff (x\in A)\wedge [(\forall C)(C\in B\implies x\in C)\wedge (\exists C)(C\in B)]\iff (\forall C)(x\in A\wedge x\in C)\wedge (\exists C)(C\in B)\iff \bigcap_{C \in B}(A \times C).$
\end{proof}

%------

%---------------------------
\bibliographystyle{apa}
\bibliography{referencias.bib}

\end{document}