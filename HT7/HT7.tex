\documentclass[a4paper,12pt]{article}
\usepackage[top = 2.5cm, bottom = 2.5cm, left = 2.5cm, right = 2.5cm]{geometry}
\usepackage[T1]{fontenc}
\usepackage[utf8]{inputenc}
\usepackage{multirow} 
\usepackage{booktabs} 
\usepackage{graphicx}
\usepackage[spanish]{babel}
\usepackage{setspace}
\setlength{\parindent}{0in}
\usepackage{float}
\usepackage{fancyhdr}
\usepackage{amsmath}
\usepackage{amssymb}
\usepackage{amsthm}
\usepackage[numbers]{natbib}
\newcommand\Mycite[1]{%
	\citeauthor{#1}~[\citeyear{#1}]}
\usepackage{graphicx}
\usepackage{subcaption}
\usepackage{booktabs}
\usepackage{etoolbox}
\usepackage{minibox}
\usepackage{hyperref}
\usepackage{xcolor}
\usepackage[skins]{tcolorbox}
%---------------------------

\newtcolorbox{cajita}[1][]{
	 #1
}

\newenvironment{sol}
{\renewcommand\qedsymbol{$\square$}\begin{proof}[\textbf{Solución.}]}
	{\end{proof}}

\newenvironment{dem}
{\renewcommand\qedsymbol{$\blacksquare$}\begin{proof}[\textbf{Demostración.}]}
	{\end{proof}}

\newtheorem{problema}{Problema}
\newtheorem{definicion}{Definición}
\newtheorem{ejemplo}{Ejemplo}
\newtheorem{teorema}{Teorema}
\newtheorem{corolario}{Corolario}[teorema]
\newtheorem{lema}[teorema]{Lema}
\newtheorem{prop}{Proposición}
\newtheorem*{nota}{\textbf{NOTA}}
\renewcommand\qedsymbol{$\blacksquare$}
\usepackage{svg}
\usepackage{tikz}
\usepackage[framemethod=default]{mdframed}
\global\mdfdefinestyle{exampledefault}{%
linecolor=lightgray,linewidth=1pt,%
leftmargin=1cm,rightmargin=1cm,
}




\newenvironment{noter}[1]{%
\mdfsetup{%
frametitle={\tikz\node[fill=white,rectangle,inner sep=0pt,outer sep=0pt]{#1};},
frametitleaboveskip=-0.5\ht\strutbox,
frametitlealignment=\raggedright
}%
\begin{mdframed}[style=exampledefault]
}{\end{mdframed}}
\newcommand{\linea}{\noindent\rule{\textwidth}{3pt}}
\newcommand{\linita}{\noindent\rule{\textwidth}{1pt}}

\AtBeginEnvironment{align}{\setcounter{equation}{0}}
\pagestyle{fancy}

\fancyhf{}









%----------------------------------------------------------
\lhead{\footnotesize Teoría de Conjuntos}
\rhead{\footnotesize  Rudik Roberto Rompich}
\cfoot{\footnotesize \thepage}


%--------------------------

\begin{document}
 \thispagestyle{empty} 
    \begin{tabular}{p{15.5cm}}
    \begin{tabbing}
    \textbf{Universidad del Valle de Guatemala} \\
    Departamento de Matemática\\
    Licenciatura en Matemática Aplicada\\\\
   \textbf{Estudiante:} Rudik Roberto Rompich\\
   \textbf{Correo:}  \href{mailto:rom19857@uvg.edu.gt}{rom19857@uvg.edu.gt}\\
   \textbf{Carné:} 19857
    \end{tabbing}
    \begin{center}
        MM2033 - Teoría de Conjuntos - Catedrático: Nancy Zurita\\
        \today
    \end{center}\\
    \hline
    \\
    \end{tabular} 
    \vspace*{0.3cm} 
    \begin{center} 
    {\Large \bf  HT 6
} 
        \vspace{2mm}
    \end{center}
    \vspace{0.4cm}
%--------------------------
\section{Sección}
\textbf{Instrucciones: }  Resuelva los siguientes problemas: 
\begin{problema}
	Sea $A=\mathbb{Z}$, si se define una relación $\approx$ sobre $A$ tal que: $x \approx y$ si $x-y \in \mathbb{Z}$. Determine si la relación $\approx$ es: reflexiva, simétrica, transitiva, antisimétrica, de orden parcial y/o de equivalencia. Si la relación es de equivalencia construya el conjunto cociente.
	\begin{dem}
		content...
	\end{dem}
\end{problema}

\begin{problema}
	Dado X un conjunto no vacío, muestre que existe una única relación de equivalencia $\mathrm{R}$ sobre X, tal que el conjunto cociente es un conjunto unitario.
	\begin{dem}
		content...
	\end{dem}
\end{problema}

\begin{problema}
	 Demuestre que si $R$ y $S$ son relaciones de equivalencia la intersección entre ellas también es relación de equivalencia. ¿Es la unión relación de equivalencia?
	 \begin{dem}
	 	content...
	 \end{dem}
\end{problema}

\begin{problema}
	Sea $f: X \rightarrow Y$ una función sobreyectiva, con $X$ un conjunto no vacío. Defina la relación $E=\{(a, b) \mid \quad f(a)=f(b)\}$
	\begin{enumerate}
		\item Muestre que $E$ es una relación de equivalencia.
		\begin{dem}
			content...
		\end{dem}
		\item Demuestre que los conjuntos $Y$ y $E / X$ son equipotentes (i.e. existe una función biyectiva entre $Y$ y $E / X$).
		\begin{dem}
			content...
		\end{dem}
	\end{enumerate}
	
\end{problema}

\section{Sección}
\textbf{Instrucciones: }  Resuelva los siguientes ejercicios del libro de Pinter:

\subsection{Capítulo 3}
\subsubsection{Ejercicios 3.2}
\begin{problema}(Problema 2)
	Let $G$ be a relation in $A$; prove each of the following:
	\begin{enumerate}
		\item $G$ is irreflexive if and only if $G \cap I=\varnothing$.
		\begin{dem}
			Sea
			\begin{enumerate}
				\item[($\implies$)] Supóngase $G$ no es reflexiva, $\forall x\in A \ni (x,x)\not\in G$. Sea 
				\begin{align*}
					(x,y)\in G\cap I &\implies (x,y)\in G\wedge \underbrace{(x,y)}_{x=y}\in I \\
					&\implies \underbrace{(x,x)\in G}_{(\to\gets)} \wedge (x,x)\in I
				\end{align*}
				$\therefore \forall (x,y)\not\in G\cap I\implies  G\cap I = \varnothing$.
				\item[($\impliedby$)] Supóngase $G\cap I=\varnothing$. Sea 
				$(x,y)\in G$, pero como $G\cap I=\varnothing\implies I\not\subseteq G\implies \forall x\in A\ni (x,x)\not\in G\implies G$ no es reflexiva.  
			\end{enumerate}
		$\therefore G \cap I=\varnothing$.
		\end{dem}
		\item $G$ is asymmetric if and only if $G \cap G^{-1}=\varnothing$.
		\begin{dem}
			Sea
			\begin{enumerate}
				\item[($\implies$)] Supóngase $G$ es asimétrica, tal que 
				\begin{align*}
					(x,y)\in G\cap G^{-1} &\implies \underbrace{(x,y)\in G}_{\stackrel{\text{definición}}{\text{asimetría}}}\wedge (x,y)\in G^{-1}\\
					&\implies (y,x)\not\in G \wedge (y,x)\in G (\to\gets ) 
				\end{align*}
			$\therefore (x,y)\not\in G\cap G^{-1}. \therefore G\cap G^{-1}=\varnothing$. 
				
				\item[($\impliedby$)] Supóngase $G\cap G^{-1}=\varnothing$, tal que
				\begin{align*}
					(x,y)\in G \wedge (y,x)\in G &\implies (x,y)\in G \wedge (x,y)\in G^{-1}\\
					&\implies (x,y)\in G\cap G^{-1}=\varnothing\\
					&\implies (y,x)\not\in G.
				\end{align*}
			$\therefore G$ es asimétrica.
			\end{enumerate}
			$\therefore G \cap G^{-1}=\varnothing$.
	
		\end{dem}
		\item $G$ is intransitive if and only if $(G \circ G) \cap G=\varnothing$.
			\begin{dem}
			Sea
			\begin{enumerate}
				\item[($\implies$)] Sea $G$ intransitiva, tal que 
				\begin{align*}
					(x,y)\in (G\circ G)\cap G &\implies (x,y)\in (G\circ G) \wedge (x,y)\in G\\
					&\implies \underbrace{\left[\exists z \ni (x,z)\in G \wedge (z,y)\in G \right]}_{\stackrel{\text{definición}}{\text{intransitiva}}}\wedge (x,y)\in G\\
					&\implies (x,y)\not\in G \wedge (x,y)\in G (\to\gets)
				\end{align*}
			$\therefore (x,y)\not\in (G \circ G) \cap G \implies (G \circ G) \cap G =\varnothing$. 
				
				\item[($\impliedby$)] Sea $(G \circ G) \cap G=\varnothing$ tal que, 
				
				\begin{align*}
					(x,y)\in G \wedge (x,y)\in G &\implies  \left[\exists z\ni (x,z)\in G\wedge (z,y)\in G\right]\wedge (x,y)\in G\\
					&\implies (x,y)\in (G\circ G)\wedge (x,y)\in G\\
					&\implies (x,y)\in (G \circ G) \cap G =\varnothing\\
					&\implies (x,y)\not\in G
				\end{align*}
				$\therefore G$ es intransitiva. 
			\end{enumerate}
			$\therefore (G \circ G) \cap G=\varnothing$.
			
		\end{dem}
	\end{enumerate}
\end{problema}

\begin{problema}(Problema 3)
	Show that if is an equivalence relation in $A$, then $G\circ G=G$.
	\begin{dem}
		Supóngase que tenemos una relación de equivalencia en $A$ (i.e. reflexivo, simétrico y transitivo) tal que, 
		\begin{enumerate}
			\item[($\implies$)] Sea \begin{align*}
				(x,y)\in G\circ G &\implies \exists z\ni \underbrace{(x,z)\in G \wedge (z,y)\in G}_{\stackrel{\text{definición}}{\text{transitividad}}}\\
				&\implies (x,y)\in G
			\end{align*}
		$\therefore G\circ G\subseteq G$. 
			\item[($\impliedby$)] Sea 
			\begin{align*}
				(x,y)\in G &\implies \underbrace{(x,x)\in G} _{\stackrel{\text{definición}}{\text{reflexividad}}} \wedge (x,y)\in G\\
				&\implies \exists z=x\ni (x,z)\in G\wedge (z,y)\in G\\
				&\implies (x,y)\in G\circ G
			\end{align*}
		$\therefore G\subseteq G\circ G$. 
		\end{enumerate}
		$\therefore G\circ G=G$. 
	\end{dem}
\end{problema}
\begin{problema}(Problema 7)
	Let $G$ and $H$ be relations in $A$; suppose that $G$ is reflexive and $H$ is reflexive and transitive. Show that $G \subseteq H$ if and only if $G \circ H=H$. (In particular, this holds if $G$ and $H$ are equivalence relations.)
	\begin{dem}
		Sea $G$ reflexiva y $H$ reflexiva y transitiva. 
		\begin{enumerate}
			\item[($\implies$)] Sea $G\subseteq H$, tal que 
			\begin{enumerate}
				\item[($\implies$)] Sea
				\begin{align*}
					(x,y)\in G\circ H &\implies \exists z\ni \underbrace{(x,z)\in G}_{\text{hipótesis}}\wedge (z,y)\in H \\
					&\implies \exists z\ni \underbrace{(x,z)\in H\wedge (z,y)\in H}_{\stackrel{\text{definición}}{\text{transitividad}}}\\
					&\implies (x,y)\in H 
				\end{align*}
				$\therefore G\circ H\subseteq H$.
				
				\item[($\impliedby$)]  Sea
				\begin{align*}
					(x,y)\in H &\implies \underbrace{(x,x)\in G}_{\stackrel{\text{definición}}{\text{reflexividad}}}  \wedge(x,y)\in H \\
					&\implies \exists z=x\ni (x,z)\in G\wedge (z,y)\in H\\
					&\implies (x,y)\in G\circ H
				\end{align*}
				$\therefore H\subseteq G\circ H$. 
			\end{enumerate}
			$\therefore G\circ H=H$.  
			\item[($\impliedby$)]  Sea $G\circ H=H$, tal que 
			\begin{align*}
				(x,y)\in G &\implies (x,y)\in G  \wedge \underbrace{(y,y)\in H}_{\stackrel{\text{definición}}{\text{reflexividad}}} \\
				&\implies \exists z=y \ni (x,z)\in G \wedge (z,y)\in H\\
				&\implies \underbrace{(x,y)\in G\circ H }_{\text{hipótesis}}\\
				&\implies (x,y)\in H	
					\end{align*}
			$\therefore G\subseteq H$. 
		\end{enumerate}
	\end{dem}
\end{problema}

%---------------------------------------
\subsubsection{Ejercicios 3.3}
\begin{problema}(Problema 10)
	Suppose $f: A \rightarrow B$ is an injective function, and $\left\{A_{i}\right\}_{i \in I}$ is a partition of $A$. Prove that $\left\{\bar{f}\left(\mathrm{~A}_{\mathrm{i}}\right)\right\}_{i \in I}$ is a partition of $\bar{f}(\mathrm{~A})$.
	\begin{dem}
		content...
	\end{dem}
\end{problema}

%---------------------------------------
\subsubsection{Ejercicios 3.4}
\begin{problema}(Problema 3)
	Let $f: A \rightarrow B$ be a function and let $G$ be an equivalence relation in $B$. Prove that $\breve{f}(\mathrm{G})$ is an equivalence relation in $A$.
	\begin{dem}
		content...
	\end{dem}
\end{problema}
%---------------------------------------
\subsubsection{Ejercicios 3.5}
\begin{problema}(Problema 3)
	Let $f: A \rightarrow B$ be a function and let $G$ be an equivalence relation in $B$. Prove that $\breve{f}(\mathrm{G})$ is an equivalence relation in $A$.
	\begin{dem}
		content...
	\end{dem}
\end{problema}

%---------------------------------------
\subsection{Capítulo 4}
\subsubsection{Ejercicios 4.2}
\begin{problema}(Problema 2)
	Let $f: A \rightarrow B$ be an increasing function. If $C$ is a chain of $A$, prove that $\bar{f}(C)$ is a chain of $B$.
	\begin{dem}
		content...
	\end{dem}
\end{problema}
%---------------------------------------
\subsubsection{Ejercicios 4.3}
\begin{problema}(Problema 10)
	Let $A$ and $B$ be partially ordered classes, and let $f: A \rightarrow B$ be an isomorphism. Prove each of the following:
	\begin{enumerate}
		\item  $a$ is a maximal element of $A$ iff $f(a)$ is a maximal element of $B$.
		\begin{dem}
			content...
		\end{dem}
		\item $a$ is the greatest element of $A$ iff $f(a)$ is the greatest element of $B$.
		\begin{dem}
			content...
		\end{dem}
		\item Suppose $C \subseteq A ; x$ is an upper bound of $C$ iff $f(x)$ is an upper bound of $\bar{f}(C)$.
		\begin{dem}
			content...
		\end{dem}
		\item $b=\sup C$ iff $f(b)=\sup \bar{f}(C)$.
		\begin{dem}
			content...
		\end{dem}
	\end{enumerate}
	
\end{problema}
%---------------------------------------
\subsubsection{Ejercicios 4.5}
\begin{problema}(Problema 1)
	Let $A$ be a fully ordered set. Prove that the set of all sections of $A$ (ordered by inclusion) is fully ordered.
	\begin{dem}
		content...
	\end{dem}
\end{problema}
\begin{problema}(Problema 9)
	Let $A$ be a well-ordered class; prove the following:
	\begin{enumerate}
		\item The intersection of any family of sections of $A$ is a section of $A$.
		\begin{dem}
			content...
		\end{dem}
		\item The union of any family of sections of $A$ is a section of $A$.
		\begin{dem}
			content...
		\end{dem}
	\end{enumerate}
	
\end{problema}
%---------------------------------------
\subsubsection{Ejercicios 4.6}
\begin{problema}(Problema 4)
	Let $A$ and $B$ be well-ordered classes. Prove that if $f: A \rightarrow B$ and $g: B \rightarrow A$ are isomorphisms, then $g=f^{-1}$.
	\begin{dem}
		content...
	\end{dem}
\end{problema}



	

%---------------------------
\bibliographystyle{apa}
\bibliography{referencias.bib}

\end{document}