   \documentclass[a4paper,12pt]{article}
\usepackage[top = 2.5cm, bottom = 2.5cm, left = 2.5cm, right = 2.5cm]{geometry}
\usepackage[T1]{fontenc}
\usepackage[utf8]{inputenc}
\usepackage{multirow} 
\usepackage{booktabs} 
\usepackage{graphicx}
\usepackage[spanish]{babel}
\usepackage{setspace}
\setlength{\parindent}{0in}
\usepackage{float}
\usepackage{fancyhdr}
\usepackage{amsmath}
\usepackage{amssymb}
\usepackage{amsthm}
\usepackage[numbers]{natbib}
\newcommand\Mycite[1]{%
	\citeauthor{#1}~[\citeyear{#1}]}
\usepackage{graphicx}
\usepackage{subcaption}
\usepackage{booktabs}
\usepackage{etoolbox}
\usepackage{minibox}
\usepackage{hyperref}
\usepackage{xcolor}
\usepackage[skins]{tcolorbox}
%---------------------------

\newtcolorbox{cajita}[1][]{
	 #1
}

\newenvironment{sol}
{\renewcommand\qedsymbol{$\square$}\begin{proof}[\textbf{Solución.}]}
	{\end{proof}}

\newenvironment{dem}
{\renewcommand\qedsymbol{$\blacksquare$}\begin{proof}[\textbf{Demostración.}]}
	{\end{proof}}

\newtheorem{problema}{Problema}
\newtheorem{definicion}{Definición}
\newtheorem{ejemplo}{Ejemplo}
\newtheorem{teorema}{Teorema}
\newtheorem{corolario}{Corolario}[teorema]
\newtheorem{lema}[teorema]{Lema}
\newtheorem{prop}{Proposición}
\newtheorem*{nota}{\textbf{NOTA}}
\renewcommand\qedsymbol{$\blacksquare$}
\usepackage{svg}
\usepackage{tikz}
\usepackage[framemethod=default]{mdframed}
\global\mdfdefinestyle{exampledefault}{%
linecolor=lightgray,linewidth=1pt,%
leftmargin=1cm,rightmargin=1cm,
}




\newenvironment{noter}[1]{%
\mdfsetup{%
frametitle={\tikz\node[fill=white,rectangle,inner sep=0pt,outer sep=0pt]{#1};},
frametitleaboveskip=-0.5\ht\strutbox,
frametitlealignment=\raggedright
}%
\begin{mdframed}[style=exampledefault]
}{\end{mdframed}}
\newcommand{\linea}{\noindent\rule{\textwidth}{3pt}}
\newcommand{\linita}{\noindent\rule{\textwidth}{1pt}}

\AtBeginEnvironment{align}{\setcounter{equation}{0}}
\pagestyle{fancy}

\fancyhf{}









%----------------------------------------------------------
\lhead{\footnotesize Teoría de Conjuntos}
\rhead{\footnotesize  Rudik Roberto Rompich}
\cfoot{\footnotesize \thepage}


%--------------------------

\begin{document}
 \thispagestyle{empty} 
    \begin{tabular}{p{15.5cm}}
    \begin{tabbing}
    \textbf{Universidad del Valle de Guatemala} \\
    Departamento de Matemática\\
    Licenciatura en Matemática Aplicada\\\\
   \textbf{Estudiante:} Rudik Roberto Rompich\\
   \textbf{Correo:}  \href{mailto:rom19857@uvg.edu.gt}{rom19857@uvg.edu.gt}\\
   \textbf{Carné:} 19857
    \end{tabbing}
    \begin{center}
        MM2033 - Teoría de Conjuntos - Catedrático: Nancy Zurita\\
        \today
    \end{center}\\
    \hline
    \\
    \end{tabular} 
    \vspace*{0.3cm} 
    \begin{center} 
    {\Large \bf  Ejercicio 6
} 
        \vspace{2mm}
    \end{center}
    \vspace{0.4cm}
%--------------------------

\section{Problema } 

\subsection{Sección 2.2}
\begin{problema}[Ejercicio 5]
	Sean $f: A \rightarrow B$ y $g: C \rightarrow D$ funciones. El producto de $f$ y $g$  es la función definida como: 
	$$
	[f \cdot g](x, y)=(f(x), g(y)) \text { para cada }(x, y) \in A \times C
	$$
	\begin{enumerate}
		\item Probar que $f \cdot g$ es una función de $A \times C$ a $B \times D$. 
		\begin{dem}
			A probar: $f\cdot g$ es función tal que usando la definición 
			\begin{enumerate}
				\item Sea $\forall (x,y)\in A\times C, \exists z\in B\times D\ni ((x,y),z)\in f\cdot g\implies z = 	[f \cdot g](x, y)=(f(x), g(y)) $.  
				\item Sean $z_1 = [f \cdot g](x, y)$ y $z_2= [f \cdot g](x, y)$ tales que 
				\begin{align*}
					(f(x), g(y)) &= (f(x), g(y)) \\
					[f \cdot g](x, y) &= [f \cdot g](x, y)\\
					z_1 = z_2
				\end{align*}
			\end{enumerate}
			
		\end{dem}
		\item Probar que si $f$ y $g$ son inyectivas, entonces $f \cdot g$ es inyectiva, y si $f$ y $g$ son sobreyectivas, entonces $f \cdot g$ es sobreyectiva. 
		\begin{dem}
			Probaremos inyectividad y sobreyectividad. 
			\begin{enumerate}
				\item Inyectividad. Sea 
					$[f\cdot g](a,b) = [f\cdot g](c,d)\implies(f(a),g(b))=(f(c),g(d))\implies [f(a)=f(c)]\wedge [g(b)=g(d)]\implies f$ y $g$ son inyectivas $\implies[a=c]\wedge [b=d]\implies (a,b)=(c,d)$. Por lo tanto, $f\cdot g$ es inyectiva.   
				\item Sobreyectividad. Sea $\forall z\in B\times D, \exists(x,y)\in A\times C\ni ((x,y),z)\in f\cdot g\implies z = 	[f \cdot g](x, y)=(f(x), g(y)) $.  
			\end{enumerate}
		\end{dem}
		\item Probar que $\operatorname{ran}[f \cdot g]=(\operatorname{ran} f) \times(\operatorname{ran} g)$.
		\begin{dem}
			A probar: $\operatorname{ran}[f \cdot g]=(\operatorname{ran} f) \times(\operatorname{ran} g)$ tal que 
			\begin{enumerate} 
				\item[$(\subseteq)$] $(x,y) \in \operatorname{ran}[f \cdot g]\implies (x,y)\in C\times D\implies (x\in C)\vee (y\in D)\implies $ Por la biyectividad de $f$ y $g$ $\implies x\in (\operatorname{ran} f) \vee y \in (\operatorname{ran} g)\implies (x,y)\in (\operatorname{ran} f) \times(\operatorname{ran} g)$. $\therefore \operatorname{ran}[f \cdot g]\subseteq (\operatorname{ran} f) \times(\operatorname{ran} g)$. 
				\item[$(\supseteq)$] $(x,y)\in \left[(\operatorname{ran} f) \times(\operatorname{ran} g)\right]\implies  (x\in \operatorname{ran} f) \wedge (y\in \operatorname{ran} g)\implies (\exists w \ni (w, x)\in f)\wedge (\exists z \ni (z, y)\in g)\implies ((w,x)\in A\times C)\wedge ((z,y)\in B\times D)\implies (x\in C)\wedge (y\in D)\implies (x,y)\in C\times D\implies $ Por la biyectividad $(x,y)\in\operatorname{ran}[f \cdot g]$. $\therefore  (\operatorname{ran} f) \times(\operatorname{ran} g)\subseteq  \operatorname{ran}[f \cdot g]$.
			\end{enumerate}
			
			Por lo tanto, $\operatorname{ran}[f \cdot g]=(\operatorname{ran} f) \times(\operatorname{ran} g)$.
			
		\end{dem}
	\end{enumerate}
	
\end{problema}


%---------------
\subsection{Sección 2.3}
\begin{problema}[Ejercicio 2]
	Supóngase que $f: A \rightarrow B$ y $g: B \rightarrow C$ son funciones. 
	\begin{enumerate}
		\item Probar que si $g \circ f$  es inyectiva, entonces $f$  es inyectiva. 
		\begin{dem}
			Sea $f(a)=f(b), \text{para algunos }  a,b\in A \implies $ Por hipótesis, $g(f(a))=g(f(b)) \implies a=b$. Por lo tanto, $f$ debe ser inyectiva. 
		\end{dem}
		\item Probar que si $g \circ f$  es sobreyectiva, entonces $g$  es sobreyectiva.
		\begin{dem}
			Por hipótesis, $\forall z \in C, \exists x \in A \ni z=g(f(x))$. Ahora bien, dígase que $\forall z \in C, \exists y=f(x) \in B\ni z=g(y)$.
		\end{dem}
		\item Concluir que si $g \circ f$ es biyectiva, entonces $f$  es inyectiva y $g$ es sobreyectiva.
		\begin{dem}
			Por la definición de biyectividad, entonces $g\circ f$ es inyectiva y sobreyectiva y por los dos incisos anteriores entonces $f$  es inyectiva y $g$ es sobreyectiva.
		\end{dem}
	\end{enumerate}
\end{problema}

%---------------
\begin{problema}[Ejercicio 5]
	Sean $g: B \rightarrow C$ y $h: B \rightarrow C$ son funciones. Suponga que $g \circ f=h \circ f$ para cada función $f: A \rightarrow B$. Probar que $g=h$.
\end{problema}
\begin{dem}
	Por hipótesis, $g(f(y))=h(f(y)), \forall y \in B$. Por la definición de función, $\forall y \in B, \exists x\in A \ni y=f(x)$. Por lo tanto, $(z,y)\in g \iff z= g(y)\iff z= g(f(x))\iff z= h(f(x))\iff z= h(y)\iff (z,y)\in h$. Por lo tanto, $g=h$.  
\end{dem}
\newpage
	\begin{cajita}
		Sean $f: A\to B,g:B\to C$ y $g\circ f: A\to C$ funciones.
	\end{cajita}
\section{Problema}
\begin{problema}
	Si $f$ y $g$ son inyectivas, entonces $(\mathrm{g} \circ \mathrm{f})$ es inyectiva.
\end{problema}
\begin{dem}
	Sea $g(f(a))=g(f(b))\implies f(a)=f(b)\implies a=b$. Por lo tanto, $g\circ f$ es inyectiva. 
\end{dem}
\section{Problema}
\begin{problema}
	Si $f$ y $g$ son sobreyectivas, entonces $(\mathrm{g} \circ \mathrm{f})$ es sobreyectiva. 
\end{problema}
\begin{dem}
	Por hipótesis, sean $\forall h \in B, \exists x \in A \ni h=f(x)$ y $\forall z \in C, \exists i \in B \ni z=g(i)$.  $\implies \forall z \in C, \exists x \in A \ni z=g(f(x))$. Por lo tanto, $(\mathrm{g} \circ \mathrm{f})$ es sobreyectiva. 
\end{dem}

\section{Problema}
\begin{problema}
	Si $f$ y $g$ son biyectivas, entonces $(\mathrm{g} \circ \mathrm{f})$ es biyectiva.
\end{problema}
\begin{dem}
   Por hipótesis, $f$ y $g$ son inyectivas y sobreyectivas. $\implies$ Por lo dos incisos anteriores $g\circ f$ es inyectiva y sobreyectiva. Por lo tanto,  $(\mathrm{g} \circ \mathrm{f})$ es biyectiva.
\end{dem}

\section{Problema}
Lea la sección 4 del capítulo 2 del libro Set Theory de Charles Pinter y resuelva
\subsection{Sección 2.4}
\begin{problema}[Problema 2]
	Suponga que $f: A \rightarrow B$ es una función, $C \subseteq A$ y $D \subseteq B$
	\begin{enumerate}
		\item  Si $f$  es inyectiva, probar que $C=\breve{f}[\bar{f}(C)]$.
		\begin{dem}
	Sean $f:A\to B$ tal que,   
			\begin{enumerate}
				\item[$\subseteq$]  Sea $ x\in C$, como $f$ es función $\exists y \in B\ni f(x)=y\implies f(x)\in \bar{f}(C)\implies x\in \breve{f}[\bar{f}(C)]$.
				 $\therefore C\subseteq \breve{f}[\bar{f}(C)]$. 
				\item[$\supseteq$] Sea $x\in\breve{f}[\bar{f}(C)]\implies f(x)\in \bar{f}(C) \implies \exists w \in C \ni f(x)=f(w)\implies $ Por la inyectividad $x=w\implies x\in C$. $\therefore \breve{f}[\bar{f}(C)]\subseteq C$. 
			\end{enumerate}
			 
		\end{dem}
		\item Si $f$  es sobreyectiva, probar que $D=\bar{f}[\breve{f}(\mathrm{D})]$
		\begin{dem}
			Sean $f:A\to B$ tal que,   
				\begin{enumerate}
				\item[$\subseteq$] Sea $y\in D$. Por la sobreyectividad $\exists w \in \breve{f}(\mathrm{D}) \ni  y = f(w)\implies y \in \bar{f}[\breve{f}(\mathrm{D})]$. $\therefore D\subseteq\bar{f}[\breve{f}(\mathrm{D})]$. 
				\item[$\supseteq$] Sea $x\in \bar{f}[\breve{f}(\mathrm{D})] \implies \exists w \in \breve{f}(\mathrm{D}) \ni  x = f(w)\implies f(w)\in D\ni x=f(w)\implies x\in D$. $\therefore \bar{f}[\breve{f}(\mathrm{D})]\subseteq D$. 
			\end{enumerate}
		Por lo tanto, $D=\bar{f}[\breve{f}(\mathrm{D})]$. 
		\end{dem}
	\end{enumerate}
\end{problema}
\begin{problema}[Problema 4]
	Sea $f: A \rightarrow B$ sea un función. Probar lo siguiente: 
	\begin{enumerate}
		\item  Si $f$  es inyectiva, entonces $\breve{f} \circ \bar{f}$ es biyectiva. [Hint: Usar el resultado del ejercicio 2(1).]
		\begin{dem}
			 Debemos probar que $\breve{f} \circ \bar{f}$ es inyectiva y sobreyectiva tal que: 
			\begin{enumerate}
				\item Inyectividad. Sea $\forall x, y \in \breve{f} \circ \bar{f}(x) = \breve{f} \circ \bar{f}(y)\implies \breve{f}(\bar{f}(x))= \breve{f}(\bar{f}(y))$. Conocemos del \textbf{Problema 5.1 (Problema 2.1)} que dado $C=\breve{f}[\bar{f}(C)]$, donde $C\subseteq  A$ y $f$ es inyectiva. $\implies x= y$. $\therefore \breve{f} \circ \bar{f} $ es inyectiva.  
				\item Sobreyectividad. Conocemos del \textbf{Problema 5.1 (Problema 2.1)} que $\breve{f} \circ \bar{f}(x)=x$. Sea $x\in C$, entonces por la definición de función $\exists y \in A  \ni y= \bar{f}(x)$. Por lo tanto, $\breve{f}(\bar{f}(x)) =\breve{f}(y) = x$. 
			\end{enumerate}
		\end{dem}
		\item Si $f$ es sobreyectiva, entonces $\bar{f} \circ \breve{f}$ es biyectiva. [Hint: Usar el resultado del ejercicio 2(2).]
			\begin{dem}
		 Debemos probar que $\bar{f} \circ \breve{f}$ es inyectiva y sobreyectiva tal que: 
			\begin{enumerate}
				\item Inyectividad. Sea $\forall x, y \in A \ni \bar{f} \circ \breve{f}(x)=\bar{f} \circ \breve{f}(y)\implies \bar{f} (\breve{f}(x))=\bar{f} (\breve{f}(y))$.  Conocemos del \textbf{Problema 5.1 (Problema 2.2)} que dado $D=\bar{f}[\breve{f}(\mathrm{D})]$, donde $D\subseteq B$ y $f$ es sobreyectiva.$\implies x= y$. $\therefore \bar{f} \circ \breve{f}$ es inyectiva.
				\item Sobreyectividad. Conocemos del \textbf{Problema 5.1 (Problema 2.2)} que $\bar{f} \circ \breve{f}(x)=x$. Sea $y\in D$, entonces por la definición de función $\exists x \in A\ni \breve{f}(y) = x$. Por lo tanto, $\bar{f}(\breve{f}(y)) =\bar{f}(x) = x$. 
			\end{enumerate} 
		\end{dem}
	\end{enumerate}

\end{problema}
\begin{problema}[Problema 8]
	Sea $f: A \rightarrow B$ una función. Probar que 
	$$
	\bar{f}(C \cap D)=\bar{f}(C) \cap \bar{f}(D)
	$$
	para cada par de subclases $C \subseteq A$ y $D \subseteq A$ si y solo si $f$ es inyectiva.
\end{problema}
\begin{dem}
	Sea $f$ una función, tales que 
	\begin{enumerate}
		
		\item [$(\impliedby)$ ]  A probar: $f$ inyectiva, entonces $\bar{f}(C \cap D)=\bar{f}(C) \cap \bar{f}(D)$. Probaremos la doble contención: 
		\begin{enumerate}
		\item[$(\supseteq)$] Sea $y\in \bar{f}(C \cap D)\implies \exists x \in C\cap D \ni y = f(x)\implies x\in C \wedge x\in D \ni y=f(x)\implies (x\in C\ni y=f(x))\wedge (x\in D\ni y=f(x)) \implies x\in \bar{f}(C) \cap \bar{f}(D)$. $\therefore \bar{f}(C \cap D)\subseteq \bar{f}(C) \cap \bar{f}(D)$
		\item[$(\subseteq)$] Sea $y\in \bar{f}(C) \cap \bar{f}(D) \implies y \in \bar{f}(C) \wedge y \in \bar{f}(D)\implies (\exists z\in C \ni y = f(z))\wedge (\exists w\in C\ni y = f(w))\implies (\exists z\in C)\wedge(\exists w \in C)\ni y= f(z)=f(w)\implies$ Por la inyectividad, $w=z$. $\therefore \bar{f}(C) \cap \bar{f}(D) \subseteq \bar{f}(C \cap D)$.
		
		\end{enumerate}
	Por lo tanto, $\bar{f}(C \cap D)=\bar{f}(C) \cap \bar{f}(D) $ .
		\item [$(\implies)$ ] A probar: $\bar{f}(C \cap D)=\bar{f}(C) \cap \bar{f}(D)$ entonces $f$ es inyectiva. Por reducción al absurdo, supóngase que $f$ no es inyectiva, tal que 
	$$\exists x,y \in A\ni z = f(x) = f(y) \implies x\neq y.$$
	Sin embargo, nótese que si $f$ no es inyectiva entonces no se cumpliría la contención del inciso anterior, tal que $\bar{f}(C) \cap \bar{f}(D) \not \subseteq \bar{f}(C \cap D) (\to \gets)$ ya que $\bar{f}(C \cap D)=\bar{f}(C) \cap \bar{f}(D)$ por hipótesis. Por lo tanto, $f$ es inyectiva. 
	\end{enumerate}


\end{dem}

%---------------------------
\bibliographystyle{apa}
\bibliography{referencias.bib}

\end{document}