\documentclass[a4paper,12pt]{article}
\usepackage[top = 2.5cm, bottom = 2.5cm, left = 2.5cm, right = 2.5cm]{geometry}
\usepackage[T1]{fontenc}
\usepackage[utf8]{inputenc}
\usepackage{multirow} 
\usepackage{booktabs} 
\usepackage{graphicx}
\usepackage[spanish]{babel}
\usepackage{setspace}
\setlength{\parindent}{0in}
\usepackage{float}
\usepackage{fancyhdr}
\usepackage{amsmath}
\usepackage{amssymb}
\usepackage{amsthm}
\usepackage[numbers]{natbib}
\newcommand\Mycite[1]{%
	\citeauthor{#1}~[\citeyear{#1}]}
\usepackage{graphicx}
\usepackage{subcaption}
\usepackage{booktabs}
\usepackage{etoolbox}
\usepackage{minibox}
\usepackage{hyperref}
\usepackage{xcolor}
\usepackage[skins]{tcolorbox}
%---------------------------

\newtcolorbox{cajita}[1][]{
	 #1
}

\newenvironment{sol}
{\renewcommand\qedsymbol{$\square$}\begin{proof}[\textbf{Solución.}]}
	{\end{proof}}

\newenvironment{dem}
{\renewcommand\qedsymbol{$\blacksquare$}\begin{proof}[\textbf{Demostración.}]}
	{\end{proof}}

\newtheorem{problema}{Problema}
\newtheorem{definicion}{Definición}
\newtheorem{ejemplo}{Ejemplo}
\newtheorem{teorema}{Teorema}
\newtheorem{corolario}{Corolario}[teorema]
\newtheorem{lema}[teorema]{Lema}
\newtheorem{prop}{Proposición}
\newtheorem*{nota}{\textbf{NOTA}}
\renewcommand\qedsymbol{$\blacksquare$}
\usepackage{svg}
\usepackage{tikz}
\usepackage[framemethod=default]{mdframed}
\global\mdfdefinestyle{exampledefault}{%
linecolor=lightgray,linewidth=1pt,%
leftmargin=1cm,rightmargin=1cm,
}




\newenvironment{noter}[1]{%
\mdfsetup{%
frametitle={\tikz\node[fill=white,rectangle,inner sep=0pt,outer sep=0pt]{#1};},
frametitleaboveskip=-0.5\ht\strutbox,
frametitlealignment=\raggedright
}%
\begin{mdframed}[style=exampledefault]
}{\end{mdframed}}
\newcommand{\linea}{\noindent\rule{\textwidth}{3pt}}
\newcommand{\linita}{\noindent\rule{\textwidth}{1pt}}

\AtBeginEnvironment{align}{\setcounter{equation}{0}}
\pagestyle{fancy}

\fancyhf{}









%----------------------------------------------------------
\lhead{\footnotesize Teoría de Conjuntos}
\rhead{\footnotesize  Rudik Roberto Rompich}
\cfoot{\footnotesize \thepage}


%--------------------------

\begin{document}
 \thispagestyle{empty} 
    \begin{tabular}{p{15.5cm}}
    \begin{tabbing}
    \textbf{Universidad del Valle de Guatemala} \\
    Departamento de Matemática\\
    Licenciatura en Matemática Aplicada\\\\
   \textbf{Estudiante:} Rudik Roberto Rompich\\
   \textbf{Correo:}  \href{mailto:rom19857@uvg.edu.gt}{rom19857@uvg.edu.gt}\\
   \textbf{Carné:} 19857
    \end{tabbing}
    \begin{center}
        MM2033 - Teoría de Conjuntos - Catedrático: Nancy Zurita\\
        \today
    \end{center}\\
    \hline
    \\
    \end{tabular} 
    \vspace*{0.3cm} 
    \begin{center} 
    {\Large \bf  HT 6
} 
        \vspace{2mm}
    \end{center}
    \vspace{0.4cm}
%--------------------------
\textbf{Instrucciones:} Del libro de Set Theory de Charles Pinter, resuelva los ejercicios 3, 6, 8, 9 de la sección 1.5 del Capítulo 1.

\section{Problemas}
\begin{problema}(Ejercicio 3)
	Probar el teorema 1.38. Sean $G$ y $H$ los gráficos. Si $\operatorname{ran} H \subseteq  \operatorname{dom} G$ entonces $\operatorname{dom}(G\circ H)= \operatorname{dom} H$.
\end{problema}

\begin{proof}
	Sean $G$ y $H$ gráficas. 
	
	\begin{enumerate}
		\item[$(\implies)$] Supóngase $x\in \operatorname{dom}(G\circ H) \implies \exists y \ni (x,y)\in G\circ H\implies \exists z \wedge \exists y \ni (x,z)\in H\wedge (z,y)\in G\implies \operatorname{dom}H\wedge \operatorname{ran}G\implies \operatorname{dom}H$. Por lo tanto, $\operatorname{dom}(G\circ H)\subseteq \operatorname{dom} H$
		\item[$(\impliedby)$]  Supóngase $x\in \operatorname{dom} H\implies \exists u \ni (x,u) \in H$. Por hipótesis, $(u\in \operatorname{ran} H\implies u\in \operatorname{dom} G) \iff (\exists v \ni (v,u)\in H\implies \exists w  \ni (u,w)\in G)$,   entonces tenemos $\exists u \wedge \exists w \ni (x,u)\in H \wedge (u,w)\in G\iff \exists w\ni (x,w)\in G\circ H\iff x\in \operatorname{dom}(G\circ H)\implies (x\in \operatorname{H}\implies x\in \operatorname{dom}(G\circ H))\implies\operatorname{dom}H \subseteq  \operatorname{dom}(G\circ H)$.
	\end{enumerate}
Por lo tanto, $\operatorname{dom}(G\circ H)= \operatorname{dom} H$.
\end{proof}

\begin{problema}(Ejercicio 6)
	Si $G, H, J$, y $K$  son gráficos, probar: 
	\begin{enumerate}
		\item Si $G \subseteq H$ y $J \subseteq K$, entonces $G \circ J \subseteq H \circ K$,
		\item $G \subseteq H$ si y solo si $G^{-1} \subseteq H^{-1}$.
	\end{enumerate}
\end{problema}

\begin{problema}(Ejercicio 8)
	Sean $G$ y $H$ gráficos, probar:
	\item  Si $G \subseteq A \times B$, entonces $G^{-1} \subseteq B \times A$.
	\item  Si $G \subseteq A \times B$ y $H \subseteq B \times C$, entonces $H \circ G \subseteq A \times C$.
\end{problema}

\begin{problema}(Ejercicio 9)
	Si $G$ y $H$ son gráficos, probar: 
	\begin{enumerate}
		\item $\operatorname{dom}(G \cup H)=(\operatorname{dom} G) \cup(\operatorname{dom} H)$.
		\item $\operatorname{ran}(G \cup H)=(\operatorname{ran} G) \cup(\operatorname{ran} H)$.
		\item $\operatorname{dom} G-\operatorname{dom} H \subseteq \operatorname{dom}(G-H)$.
		\item $\operatorname{ran} G-\operatorname{ran} H \subseteq \operatorname{ran}(G-H)$.
	\end{enumerate}

\end{problema}

%---------------------------
\bibliographystyle{apa}
\bibliography{referencias.bib}

\end{document}